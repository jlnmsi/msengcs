\documentclass[10pt,article,oneside]{memoir}
\usepackage{lmodern}

\usepackage{tikz}
\usepackage{verbatim}
\usetikzlibrary{arrows,shapes}

\usepackage{varioref}
\usepackage[labelfont=bf]{caption}

\usepackage{enumitem}
\PassOptionsToPackage{hyphens}{url}\usepackage{hyperref}
\hypersetup{unicode=true,
            pdfborder={0 0 0},
            colorlinks=true,
            urlcolor  = blue,
            breaklinks=true}

\usepackage{amssymb,amsmath}
\usepackage{ifxetex,ifluatex}
\usepackage{fixltx2e} % provides \textsubscript
\ifnum 0\ifxetex 1\fi\ifluatex 1\fi=0 % if pdftex
  \usepackage[T1]{fontenc}
  \usepackage[utf8]{inputenc}
\else % if luatex or xelatex
  \ifxetex
    \usepackage{mathspec}
  \else
    \usepackage{fontspec}
  \fi
  \defaultfontfeatures{Ligatures=TeX,Scale=MatchLowercase}
\fi
% use upquote if available, for straight quotes in verbatim environments
\ifxetex
\IfFileExists{upquote.sty}{\usepackage{upquote}}{}
% use microtype if available
\IfFileExists{microtype.sty}{%
\usepackage{microtype}
\UseMicrotypeSet[protrusion]{basicmath} % disable protrusion for tt fonts
}{}

\urlstyle{same}  % don't use monospace font for urls
\usepackage{longtable,booktabs}
\usepackage{graphicx,grffile}
\makeatletter
\def\maxwidth{\ifdim\Gin@nat@width>\linewidth\linewidth\else\Gin@nat@width\fi}
\def\maxheight{\ifdim\Gin@nat@height>\textheight\textheight\else\Gin@nat@height\fi}
\makeatother
% Scale images if necessary, so that they will not overflow the page
% margins by default, and it is still possible to overwrite the defaults
% using explicit options in \includegraphics[width, height, ...]{}
\setkeys{Gin}{width=\maxwidth,height=\maxheight,keepaspectratio}
\IfFileExists{parskip.sty}{%
\usepackage{parskip}
}{% else
\setlength{\parindent}{0pt}
\setlength{\parskip}{6pt plus 2pt minus 1pt}
}
\setlength{\emergencystretch}{3em}  % prevent overfull lines
\providecommand{\tightlist}{%
  \setlength{\itemsep}{0pt}\setlength{\parskip}{0pt}}
\setcounter{secnumdepth}{0}
% Redefines (sub)paragraphs to behave more like sections
\ifx\paragraph\undefined\else
\let\oldparagraph\paragraph
\renewcommand{\paragraph}[1]{\oldparagraph{#1}\mbox{}}
\fi
\ifx\subparagraph\undefined\else
\let\oldsubparagraph\subparagraph
\renewcommand{\subparagraph}[1]{\oldsubparagraph{#1}\mbox{}}
\fi

\setromanfont[Mapping=tex-text,BoldFont={Minion Pro Bold}]{Minion Pro}
\setsansfont[Mapping=tex-text]{Minion Pro}
\setmonofont[Mapping=tex-text,Scale=MatchLowercase]{PragmataPro}
%\setmonofont[Mapping=tex-text,Scale=MatchLowercase,BoldFont={Consolas Bold}]{Consolas}


\settrimmedsize{11in}{210mm}{*}
 \setlength{\trimtop}{0pt}
 \setlength{\trimedge}{\stockwidth}
 \addtolength{\trimedge}{-\paperwidth}
 \settypeblocksize{7.75in}{33pc}{*}
 \setulmargins{4cm}{*}{*}
 \setlrmargins{1.25in}{*}{*}
 \setmarginnotes{17pt}{51pt}{\onelineskip}
 \setheadfoot{\onelineskip}{2\onelineskip}
 \setheaderspaces{*}{2\onelineskip}{*}
 \checkandfixthelayout

  \makepagestyle{myruled}
 %   \makeheadrule {myruled}{\textwidth}{\normalrulethickness}
    \makeevenhead {myruled}{} {} {\itshape\rightmark}
    \makeoddhead  {myruled}{\itshape\leftmark} {} {}
    \makeevenfoot {myruled}{}{\thepage}    {}
    \makeoddfoot  {myruled}{}{\thepage}    {}
    \makeatletter % because of \@chapapp
    \makepsmarks  {myruled}{
    \nouppercaseheads
    \createmark {section} {left} {nonumber} {} {}
    \createmark {subsection} {right} {nonumber} {} {}


%     \createmark {subsubsection} {left} {shownumber} {} {}
   }
   \makeatother
   \pagestyle{myruled}
\fi
\counterwithout{section}{chapter}
  \maxsecnumdepth{chapter} 
  \setsecnumdepth{chapter}
\setcounter{secnumdepth}{3}

\begin{document}

\section{Aspektområde: Miljö, resurs och område}
\subsection{Aspekt: Yrkesexamen}

\emph{Angiven examen är reglerad och ryms inom examensordningen.}

\emph{Utbildningens innehåll inklusive eventuella inriktningar har rimlig omfattning och avgränsning i förhållande till yrkesexamen.}

\emph{I ett rikstäckande perspektiv finns ett allmänt intresse av att examen för utfärdas.}

\begin{itemize}    
\item Ange examen (examensbenämning inklusive ev. inriktningar som utexaminerade studenter ska få).
\item Beskriv översiktligt utbildningens omfattning innehåll inklusive inriktningar. Bifoga en översikt över utbildningens struktur.
\item Beskriv och analysera utbildningens omfattning och innehåll i förhållande till den vetenskapliga/ konstnärliga grunden och vetenskaplig bredd och djup.
\item Beskriv varför lärosätet vill erbjuda en utbildning som leder till aktuell yrkesexamen, vilket behov utbildningen täcker i relation till samhälle och befintligt regionalt och nationellt utbildningsutbud. Ange också när starten är planerad och hur stor antagning av studenter som är avsedd.
\end{itemize}  

\subsection{Aspekt: Personal (Lärarkompetens och lärarkapacitet)}

\emph{Antalet lärare och deras sammantagna kompetens är adekvat och står i proportion till utbildningens omfattning, innehåll, storlek och genomförande.}

\begin{itemize}    
\item Beskriv (se separat lärartabell grundnivå och avancerad nivå), lärarnas kompetens (vetenskapliga/konstnärliga/pedagogiska/professionsrelaterade) och förklara varför den är tillräcklig och adekvat och står i proportion till utbildningens planerade undervisning, handledning och examination.
\item Beskriv och analysera hur arbetet sker långsiktigt för att säkerställa att det finns tillräckliga lärarresurser.
\item Beskriv och analysera lärarnas utrymme för kompetensutveckling och hur förutsättningar skapas för lärarnas kompetensutveckling både individuellt och kollegialt.
\end{itemize}  

\subsection{Aspekt: Utbildningsmiljön}

\emph{Det finnns en för utbildningen relevant vetenskaplig och professionsinriktad miljö.}

\emph{Relevant samverkan sker med det omgivande samhället.}
    
\begin{itemize}    
\item Beskriv och analysera utbildningens vetenskapliga/konstnärliga miljö och på vilket sätt verksamheten bedrivs så att det finns nära samband och anknytning mellan forskning och utbildning
\item Beskriv och analysera på vilket sätt studenterna, inklusive eventuella distansstudenter eller studenter på annan ort, genom utbildningen får delta i ett forskande sammanhang och tillägna sig ett forskande förhållningssätt.
\item Beskriv och analysera samverkan med det omgivande samhället och på vilket sätt den kommer att ge konkret avtryck för studenterna i utbildningen.
\end{itemize}  

\subsection{Aspekt: Resurser}

\emph{Det finns tillgång till en stabil och ändamålsenlig infrastruktur.}

\emph{De tillgängliga resurserna utnyttjas effektivt för att hålla en hög kvalitet i verksamheten.}
    
\begin{itemize}
\item Beskriv studenternas tillgång till litteratur och annat lärandematerial, informationstekniska resurser samt infrastruktur i övrigt som krävs för att de ska kunna tillgodogöra sig utbildningen på ett relevant sätt. (Behoven av infrastruktur varierar beroende på vilken utbildning det handlar om. För utbildningar med experimentella inslag är t.ex. tillgång till goda laborativa förhållanden nödvändig och för konstnärliga utbildningar kan det handla om t.ex. verkstäder och repetitionslokaler. För distansutbildningar kan det handla om att det bör finnas välutvecklade kommunikationsformer.)
\item Beskriv och analysera arbetet med att säkerställa att tillgängliga resurser utnyttjas effektivt för att hålla en hög kvalitet i verksamheten.
\end{itemize}

\section{Aspektområde: Utformning, genomförande och resultat}

\subsection{Aspekt: Styrdokument (utbildningsplan och kursplaner)}

\emph{Det finns utbildningsplan och kursplaner för hela utbildningen.}
    
\begin{itemize}
\item Bifoga styrdokument, dvs. utbildningsplan och kursplaner, för hela utbildningen, där det av kursplanerna bör framgå hur undervisning, kurslitteratur/annat lärandematerial och examination kommer att vara utformade.
\item Beskriv hur dessa styrdokument fastslås, förnyas och kvalitetssäkras.
\end{itemize}  

\subsection{Aspekt: Säkring av examensmålen}

\emph{Genom utbildningens utformning, genomförande och examination säkerställs att studenterna uppnått målen i examensordningen när examen utfärdas (särskild för aktuell examen).}

\begin{itemize}    
\item Beskriv och analysera hur utbildningens utformning och genomsförande säkerställer att studenterna uppnår examensmålen
\item Beskriv och analysera hur utbildningens utformning visar på en progression och koppling mellan examensmål, lärandemål, lärandeaktiviteter och examination.
\item Beskriv och analysera hur utbildningens utformning och genomförande främjar studenternas lärande och tar hänsyn till studenternas förutsättningar.
\end{itemize}

\section{Arbetslivets perspektiv}

\emph{Utbildningen är användbar och förbereder studenter för ett föränderligt arbetsliv.}
  
\begin{itemize}  
\item Beskriv och analysera hur utbildningens innehåll och utformning säkerställer användbarhet och förberedelse för arbetslivet.
\item Beskriv och analysera hur information inhämtas som är relevant för utbildningens kvalitetssäkring och utveckling med hänsyn till dess användbarhet och förberedelse för arbetslivet.
\end{itemize}

\section{Studenters perspektiv}

\emph{Utbildningen verkar för att studenterna tar en aktiv del i arbetet med att utveckla utbildningen.}
  
\begin{itemize}  
\item Beskriv systemet för att säkra studentin ytande och hur det dokumenteras.
\item Beskriva hur utfall av kursvärderingar och eventuella åtgärder sammanställs och återkopplas till studenterna.
\item Beskriv och analysera arbetet för att studenterna ska ta aktiv del och i dialog med lärarna utveckla utbildningens samtliga delar.
\end{itemize}  

\section{Jämställdhetsperspektiv}

\emph{Jämställdhetsperspektiv är integrerat i utbildningens utformning och genomförande.}
  
\begin{itemize}  
\item Beskriv och analysera hur arbetet sker för att jämställdhetsintegrera utbildningens innehåll och utformning, exempelvis med avseende på kursplaner, studentlitteratur och kommande studentpopulation.
\end{itemize}  

\section{Utbildningsplan}

\section{Kursplaner}

\end{document}